
\chapter{Related Literature and Studies}
\begin{refsection}

This chapter presented a review of both international and local literature relevant to the research topic. The researchers collected information using the college library, the internet, and various other references that assisted them in their study.

\subsection{Helmet Detection using Computer Vision}
Helmet detection using computer vision involved automatically identifying helmet use among motorcyclists through AI and image processing. Using deep learning models like YOLO and tools like OpenCV, this prototype detected violations in real-time. It enhanced road safety, supported law enforcement, and reduced manual monitoring in traffic surveillance applications.

According to \citeauthor{afzal2021helmet} [\citeyear{afzal2021helmet}] developed a deep learning-based automatic helmet detection system for real-time videos. The researchers used various models, with Faster R-CNN and Region Proposal Network (RPN) addressing challenges like low resolution, weather, occlusion, and illumination. The model was retrained on a self-generated dataset from three locations in Lahore, Pakistan. The system achieved an impressive accuracy of 97.26\%, demonstrating its potential to improve authorities' ability to monitor motorcyclists violating traffic laws \cite{afzal2021helmet}. Another review from \citeauthor{singh2024visual} [\citeyear{singh2024visual}] emphasizes how computer vision improves helmet compliance in dangerous industries by using AI-powered detection to monitor safety in real-time, enforce compliance, and reduce workplace dangers \cite{singh2024visual}. Furthermore, \citeauthor{siebert2024urban} [\citeyear{siebert2024urban}] developed a low-cost and affordable computer vision method to check if helmets are used by motorcycle riders in five cities in Southeast Asia using crowdsourced images from Mapillary. They trained their algorithm on over 800,000 images and it achieved high accuracy and detected over 1.3 million motorcycles. The results show that drivers are more often to wear helmets than passengers and people wear helmets more on big roads than small roads. This approach is helpful because it is accurate and useful without the need for people to go out and collect data \cite{siebert2024urban}. 

According to \citeauthor{giron2020motorcycle} [\citeyear{giron2020motorcycle}] the no helmet no ride law that was implemented in the Philippines is still not working because many motorcycle riders are not following it. The government has partnered with De La Salle University to fix this issue by using artificial intelligence. The reseachers used Computer Vision to automatically check if riders are wearing helmets or not. The system used deep learning, especially Convolutional neural networks to improve detection accuracy \cite{giron2020motorcycle}. According to \citeauthor{soltanikazemi2023helmet} [\citeyear{soltanikazemi2023helmet}] , the helmet violation detection system was YOLOv5-based and it is developed using genetic algorithm optimization to monitor in real-time. The model achieved a high accuracy, and it was ranked 4th in the AI City Challenge in 2023. The study shows the advantages of deep learning and discusses how it is more effective than the traditional methods in ensuring helmet compliance and enhancing motorcycle safety in roads \cite{soltanikazemi2023helmet}. 

The work of \citeauthor{mutyala2023helmet} [\citeyear{mutyala2023helmet}], introduced a real-time helmet detection warning system that is powered by Detection Transformers (DETR) for improving detection precision and operational effectiveness in improving motorcycle safety. The system detects motorcycle riders that are not using and wearing helmets in real time, and it will generate alerts to improve the road safety. DETR uses a self-attention mechanism to capture complex relationships in image sequences, allowing accurate helmet detection even in difficult conditions like poor lighting. The system combines video feed analysis with DETR's object detection features, ensuring minimal processing delay. Testing results show the system's high precision and recall rates in different situations. This solution can be customized to send alerts to authorities or directly notify and inform riders, this might decrease violations and promote safety \cite{mutyala2023helmet}.  \citeauthor{tomas2023motorcycle} [\citeyear{tomas2023motorcycle}], used the YOLOv5 algorithm in their study, YOLOv5 is used to detect helmets, and it classifies their usage among motorcycle riders in the Philippines. The researchers processed video footage from Makati City and optimized the model hyperparameters for better accuracy. The study suggested enhancing data consistency and using separate models for detection and classification tasks. The findings showed the best results in helmet detection, which could contribute more to road safety \cite{tomas2023motorcycle}. 

\subsection{Object Detection Models and Image Processing Techniques}

Object detection models like YOLO are widely used for real-time detection tasks. YOLO, combined with OpenCV for image processing and TensorFlow for model training and deployment, offers efficient and accurate object recognition. These tools work together in applications such as traffic monitoring, helmet detection, and safety compliance systems.  Image processing techniques are fundamental in preparing and enhancing visual data for analysis by machine learning models. In this study, several image processing methods were applied to ensure that the system accurately detects whether an individual is wearing a helmet. These techniques help in improving the quality of the input data, extracting relevant features, and enabling more accurate and efficient model training.

The YOLO (You Only Look Once) is a deep learning-based object detection model that can detect multiple objects in real-time with high speed and accuracy. According to the study of \citeauthor{jiang2022review} [\citeyear{jiang2022review}] , the YOLO algorithm was improving and evolving from time to time, and it makes object detection faster and more accurate. It compares different YOLO versions and explains its performance compared to traditional methods like Convolutional Neural Networks (CNNs). They highlight that YOLO is still improving and evolving and it is very useful in areas like in security, finance, and other applications \cite{jiang2022review}. Similarly \citeauthor{terven2022comprehensive} [\citeyear{terven2022comprehensive}] discuss the evolution of YOLO from its first version to YOLOv8, YOLO-NAS, and YOLO with transformers. It highlights its improvements in architecture, accuracy, and speed. They compare YOLO with different models like R-CNN and SSD and they explore its applications in fields of robotics, healthcare, security, and traffic monitoring. Future research aims to improve YOLO’s real-time detection and its efficiency \cite{terven2022comprehensive}. YOLO (You Only Look Once) was introduced by Joseph Redmon and his team in 2015 to address the limitations of earlier object detection models like Fast R-CNN. While Fast R-CNN was accurate, it was too slow for real-time applications, taking 2–3 seconds to process a single image. In contrast, YOLO performs detection with just one forward pass through the network, enabling much faster and real-time predictions \cite{gfg2021yolo}.
With the increasing number of motorcycle users and the issue of helmet non-compliance, \citeauthor{kumar2024realtime}[\citeyear{kumar2024realtime}] developed a Real-Time Helmet Detection System to improve road safety by detecting helmet violations and capturing vehicle license plates. The system uses YOLO for object detection and a mechanism for license plate recognition, consisting of three steps: identifying motorcyclists, verifying helmet usage, and capturing the license plate. It achieved 64\% accuracy in vehicle identification, 78\% in helmet detection, and 92\% in license plate recognition \cite{kumar2024realtime}. In Addition, \citeauthor{muhammad2024helmet} [\citeyear{muhammad2024helmet}] designed a real-time helmet detection system using YOLOv8, deployed on edge devices to enhance the safety of motorcyclists in Indonesia. During testing, the model demonstrated strong performance in detecting helmets (91.1\% F1 score), riders (81.7\%), and non-helmeted riders (33.0\%). They also evaluate the system’s CPU usage (78\%), RAM (77.4\%), temperature (33°C–65°C), and power consumption (6.5 W). This system shows potential for integration into smart city infrastructure, improving the efficiency of traffic law enforcement \cite {muhammad2024helmet}. Furthermore,\citeauthor{choubey2025helmet} [\citeyear{choubey2025helmet}] introduces a YOLOv3-oriented model created for identifying license plates and helmets within images. They improved data quality and variety by pre-processing and created a tailored annotated dataset for helmets and license plates. The model underwent training through a multi-phase approach \cite{choubey2025helmet}.

OpenCV is an open-source library that provides a vast collection of tools for computer vision tasks, including image processing, feature extraction, object recognition, and real-time video analysis. According to \citeauthor{satheesh2024automated} [\citeyear{satheesh2024automated}] using OpenCV, a publicly available computer vision library. For the goal of observing objects, attributes for extraction and image preprocessing, OpenCV offers a broad array of tools and functionalities. By utilizing OpenCV, we will guarantee consistency and reliability when managing various real-world situations, encompassing various lighting situations, obstructions, and vehicle angles \cite{satheesh2024automated}.


\subsubsection{Vehicle Classification}
According to \citeauthor{chandrika2020vehicle} [\citeyear{chandrika2020vehicle}], the growing number of vehicles exceeding 1 billion globally makes it difficult for authorities to manage traffic and provide sufficient infrastructure. Their study introduces a vehicle detection and classification system using image processing, broken into six stages: image acquisition, analysis, object detection, counting, classification, and result display. The proposed system helps monitor traffic flow, detect rule violations, and classify vehicles into categories such as motorcycles, cars, vans, and trucks, thereby supporting better traffic planning and management \cite{chandrika2020vehicle}.

Similarly, \citeauthor{ong2022vehicle} [\citeyear{ong2022vehicle}], vehicle classification plays a key role in enhancing security, managing traffic congestion, and preventing road accidents. One challenge in this process is the poor image quality from video sources, which makes object recognition difficult. To address this, their study implemented and compared YOLOv5 and Faster R-CNN algorithms for classifying vehicles into five categories: motorcycle, car, van, bus, and lorry. The results showed that YOLOv5 outperformed Faster R-CNN, achieving a mean average precision (mAP) of 0.91, precision of 0.81, and recall of 0.86, making it more suitable for accurate vehicle classification using video-based image data \cite{ong2022vehicle}.

In line with this,\citeauthor{Sanjana2021} [\citeyear{Sanjana2021}], vehicle detection and classification have become increasingly important due to the growing number of vehicles, traffic violations, and road accidents. Their review explores various methodologies that have evolved over the years, shifting from basic image processing to machine learning approaches. This progression has led to the integration of helmet detection and license plate recognition, using object detection and text recognition models that are now easier to implement through built-in frameworks or customizable tools \cite{Sanjana2021}.

Moreover, \citeauthor{Espinosa2021} [\citeyear{Espinosa2021}] , motorcycles are classified as Vulnerable Road Users (VRUs), alongside bicycles and pedestrians, and are among the most frequently involved in urban traffic accidents. To address this issue, their study reviews the use of automatic video processing techniques particularly leveraging CCTV surveillance systems for the detection and tracking of motorcycles. The authors emphasize the effectiveness of deep learning algorithms within the field of computer vision for these tasks. Additionally, they discuss the use of standard performance metrics, introduce the Urban Motorbike Dataset (UMD) for evaluation purposes, and outline current challenges and potential future research directions in this emerging field \cite{Espinosa2021}.

\subsubsection{Passenger Counting}
Passenger counting systems utilize sensors and computer vision models to automatically count individuals boarding or exiting vehicles. These prototypes often use YOLO for real-time detection and OpenCV for image processing. They help optimize public transport operations, monitor capacity, and improve service efficiency in buses, trains, and other mass transit systems.

The study of \citeauthor{rendon2023passenger} [\citeyear{rendon2023passenger}], which introduced a computer vision method using deep learning to detect, count and estimate the number of passengers in Bogota's TransMilenio stations, this study shows how accurate passenger counting in public transport systems is important. They analyzed images with nearly 900,000 labeled heads and achieved a very accurate result, with an error of only one person per image. This is better than counting them by hand. This method is scalable and low-cost, and it is useful for improving the planning and running of public transport systems \cite{rendon2023passenger}. The paper by \citeauthor{radovan2024passenger} [\citeyear{radovan2024passenger}] discusses different passenger counting systems, comparing traditional technologies like RFID and infrared sensors with newer methods using image processing and machine learning. It explores the advantages and disadvantages of each system and how to improve these. It also discusses concerns under GDPR. The authors propose some improvements for passenger counting solutions and suggest ways to enhance public transport operations to make it more effective \cite{radovan2024passenger}.

According to the study by \citeauthor{bhatt2024ai} [\citeyear{bhatt2024ai}] , wearing a helmet when motorcycling is important because it helps reduce the likelihood of serious head injuries in accidents. With the help of modern technology such as real-time surveillance and computer vision, it is now possible to automatically determine whether riders are wearing a helmet using video footage on the road. The aim of this system is to strengthen the implementation of road safety laws by detecting not only the driver but also the passenger if they are wearing a helmet. Based on a report by the World Health Organization (WHO) in 2023, the correct use of a helmet reduces the risk of death by 42\% and the risk of head injury by 69\% [WHO, 2023] \cite{bhatt2024ai}.

\subsubsection{Evaluation Metrics of YOLOv8}
YOLO (You Only Look Once) is a real-time object detection algorithm that offers a faster and more efficient alternative to traditional detection methods. Specifically, as a single-stage detector, YOLO employs a convolutional neural network (CNN) to predict both bounding boxes and object classes directly from input images. It achieves this by dividing the image into a grid, which enables the detection of multiple objects in a single pass \cite {kili2023yolo}. 

In this context, several studies have evaluated the performance of YOLO and its variants in terms of speed, accuracy, and adaptability. According to \citeauthor{prakash2024study} [\citeyear{prakash2024study}], YOLO revolutionized object detection by enabling real-time performance through its single-pass, grid-based prediction approach \cite{prakash2024study}. Moreover,\citeauthor{karthika2024automated} [\citeyear{karthika2024automated}] assessed YOLOv8 for its high precision and speed, highlighting its effectiveness across static images, video streams, and live feeds \cite{karthika2024automated}. Furthermore,\citeauthor{Varghese2024} [\citeyear{Varghese2024}] demonstrated that YOLOv8 outperforms earlier versions by integrating attention mechanisms, dynamic convolution, and voice recognition, which results in improved accuracy and computational efficiency \cite{Varghese2024}. Similarly, \citeauthor{Safaldin2024} [\citeyear{Safaldin2024}] proposed an enhanced YOLOv8 model tailored for detecting moving objects in dynamic environments. Through architectural and preprocessing modifications, their model improved motion sensitivity and achieved strong results on datasets such as KITTI, LASIESTA, PESMOD, and MOCS recording 90\% accuracy, 90\% mAP, 30 FPS, and 80\% IoU \cite{Safaldin2024}. However,\citeauthor{hussain2024yolo} [\citeyear{hussain2024yolo}] conducted a comparative analysis of YOLO architectures, noting that YOLOv8 features enhanced feature extraction and anchor-free detection, while YOLOv10 achieves even greater real-time performance by incorporating large-kernel convolutions and eliminating non-maximum suppression \cite{hussain2024yolo}.

     
\section{Synthesis of the State-of-the-Art}
The reviewed literature highlighted the growing importance and effectiveness of computer vision and deep learning techniques in addressing road safety concerns, particularly in enforcing helmet compliance, and counting passengers in real-time.

International and local studies consistently emphasize the role of helmet detection systems using deep learning models such as YOLO (You Only Look Once), Faster R-CNN, and Detection Transformers (DETR). \citeauthor{afzal2021helmet} [\citeyear{afzal2021helmet}] demonstrated a highly accurate system using Faster R-CNN, achieving 97.26\% accuracy despite challenges like occlusion and weather conditions \cite{afzal2021helmet}. Complementing this, \citeauthor{singh2024visual} [\citeyear{singh2024visual}] and \citeauthor{giron2020motorcycle} [\citeyear {giron2020motorcycle}] recognized the potential of AI in improving helmet compliance in both traffic and industrial settings \cite{singh2024visual}, \cite{giron2020motorcycle}. Further innovations were noted by \citeauthor{siebert2024urban} [\citeyear{siebert2024urban}] , who employed crowdsourced images for a low-cost helmet detection approach, and by \citeauthor{soltanikazemi2023helmet} [\citeyear{soltanikazemi2023helmet}]  , whose YOLOv5-based system earned top ranks in the AI City Challenge. \citeauthor{mutyala2023helmet} [\citeyear{mutyala2023helmet}] introduced DETR-powered real-time systems with alert features, while \citeauthor{tomas2023motorcycle} [\citeyear{tomas2023motorcycle}] highlighted the importance of model optimization in improving helmet detection in the Philippines \cite{siebert2024urban}, \cite{soltanikazemi2023helmet}, \cite{mutyala2023helmet}, \cite{tomas2023motorcycle}.

The integration of object detection and image processing tools YOLO, OpenCV, TensorFlow has been fundamental in enabling real-time, accurate, and resource-efficient systems. \citeauthor{jiang2022review} [\citeauthor{jiang2022review}] and \citeauthor{terven2022comprehensive} [\citeyear{terven2022comprehensive}] chronicled the evolution of YOLO from its initial versions to YOLOv8 and YOLO-NAS, highlighting architectural improvements and broader applications across various fields \cite{jiang2022review}, \cite{terven2022comprehensive}. 

YOLO (You Only Look Once) is a real-time object detection algorithm that efficiently detects multiple objects in a single pass using a grid-based CNN approach. Studies highlight its speed, accuracy, and adaptability, especially in its latest version, YOLOv8.\citeauthor{prakash2024study} [\citeyear{prakash2024study}] emphasized its real-time performance, while \citeauthor{karthika2024automated} [\citeyear{karthika2024automated}] noted its high precision in various visual inputs \cite{prakash2024study}, \cite{karthika2024automated}.\citeauthor{Varghese2024} [\citeyear{Varghese2024}] showed improvements in YOLOv8 through added attention mechanisms and voice recognition, and \citeauthor{Safaldin2024} [\citeyear{Safaldin2024}] reported strong performance in dynamic environments \cite{Varghese2024}, \cite{Safaldin2024}.\citeauthor{hussain2024yolo} [\citeyear{hussain2024yolo}] compared YOLO versions, noting YOLOv8’s enhanced detection and YOLOv10’s improved performance using large-kernel convolutions and anchor-free techniques \cite{hussain2024yolo}.\citeauthor{kumar2023vision} [\citeyear{kumar2023vision}] and \citeauthor{muhammad2024helmet} [\citeyear{muhammad2024helmet}] demonstrated real-world implementations that integrate YOLO yielding high recognition rates and strong performance on edge devices \cite{kumar2023vision}, \cite{muhammad2024helmet}. Similarly, \citeauthor{choubey2025helmet} [\citeyear{choubey2025helmet}] emphasized dataset preparation and multi-phase training in developing a YOLOv3 model for detecting helmets and plates \cite{choubey2025helmet}.

The utility of OpenCV was established by \citeauthor{satheesh2024automated} [\citeyear{satheesh2024automated}] , who showed how it aids in preprocessing, feature extraction, and robustness in varied real-world conditions \cite{satheesh2024automated}. In tandem to emerged as a powerful framework for training and deploying models, with \citeauthor{kumar2024realtime} [\citeyear{kumar2024realtime}] and \citeauthor{sharma2024vision} [\citeyear{sharma2024vision}] showcasing its flexibility in creating cost-effective and scalable safety monitoring systems. \cite{kumar2024realtime} \cite{sharma2024vision}

Several studies highlight the importance of vehicle classification in improving traffic management, security, and accident prevention. \citeauthor{Espinosa2021} [\citeyear{Espinosa2021}] focus on tracking motorcycles using deep learning, while \citeauthor{Sanjana2021} [\citeyear{Sanjana2021}] highlight the shift from traditional image processing to integrated helmet and plate detection using modern frameworks \cite{Espinosa2021}, \cite{Sanjana2021}.  \citeauthor{ong2022vehicle} [\citeyear{ong2022vehicle}] show YOLOv5's superior accuracy over Faster R-CNN, and \citeauthor{chandrika2020vehicle} [\citeyear{chandrika2020vehicle}] propose a full system for detecting, counting, and monitoring vehicles \cite{ong2022vehicle}, \cite{chandrika2020vehicle}. Despite varying approaches, all studies support the effectiveness of computer vision in traffic-related applications.

The literature also covers passenger counting systems, crucial for optimizing transport services. \citeauthor{rendon2023passenger} [\citeyear{rendon2023passenger}] developed a deep learning method for head counting in Bogotá, yielding minimal errors and proving useful for transit planning \cite{rendon2023passenger}. \citeauthor{radovan2024passenger} [\citeyear{radovan2024passenger}] compared traditional methods like RFID with modern image processing approaches, offering improvements under regulatory frameworks such as GDPR \cite{radovan2024passenger}. \citeauthor{bhatt2024ai} [\citeyear{bhatt2024ai}] expanded on this by developing a real-time system that also includes helmet detection for both drivers and passengers, echoing the WHO’s findings on the life-saving importance of helmets \cite{bhatt2024ai}.

In summary, the reviewed studies underscore a significant trend: AI-powered computer vision prototypes are revolutionizing public safety enforcement. Tools like YOLO, TensorFlow and OpenCV when integrated with real-time video analysis form the backbone of intelligent traffic monitoring systems. These systems not only automate compliance checks but also promise scalability, cost-efficiency, and broad applicability in smart city infrastructures.

\section{Gap Bridged of the Study}
The existing helmet detection systems mostly focused on identifying if the motorcycle driver was wearing a helmet, often ignoring the passenger. These systems were usually made for controlled or international settings and did not consider the real traffic conditions in the Philippines, such as poor lighting, blurry movements, and blocked views in live road situations. Many of these systems also checked only if a helmet was present, without checking if it was worn properly or securely fastened. In some cases, riders just carried the helmet instead of wearing it, and these systems could not tell the difference. Also, most existing systems did not check if the rider was using the correct helmet type for the motorcycle. Another issue was that these systems did not detect overloading, where more than two people were riding a motorcycle, something that was common but often ignored.

To address these problems, this study presented a real-time helmet compliance monitoring prototype designed specifically for Philippine roads like the Nabua Highway. The prototype used the YOLOv8 object detection model to detect if helmets were being worn correctly by both drivers and passengers, checked if the helmet matched the type of vehicle, counted the number of riders to spot overloading, and saved short video clips of violations as evidence. This offered a more complete, localized, and practical way to support traffic law enforcement and improve road safety.









%=======================================================%
%%%%% Do not delete this part %%%%%%
\clearpage

%\printbibliography[heading=subbibintoc, title={\texorpdfstring{\centering}{} Notes}]
\printbibliography[
heading=subbibintoc,
title={\texorpdfstring{Notes}{Notes}}
]

\end{refsection}